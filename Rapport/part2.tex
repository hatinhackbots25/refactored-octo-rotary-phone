\section{Analyse mathématique des systèmes \textit{SCEX\_T1} et \textit{SCEX\_T2}}
\subsection{\textit{SCEX\_T1}}
\subsubsection{Decimation cipher}
Ici, nous allons pouvoir retrouver une méthode de chiffrement dite de \enquote{\textit{décimation}}. Cette dernière se présente sous la forme suivante :
\begin{center}
    \begin{equation}
	c_t = \left\{
	    \begin{array}{lll}
		m_{t}\oplus\sigma_{t} & \mbox{si } t\equiv 0 & (\mbox{mod } 3) \\
		m_{t}\oplus\sigma_{t+1} & \mbox{si } t\equiv 1 & (\mbox{mod } 3) \\
		m_{t}\oplus\sigma_{t+2} & \mbox{si } t\equiv 2 & (\mbox{mod } 3)
	    \end{array}
	\right.
	\label{scext1}
    \end{equation}
\end{center}

Le fonctionnement est très simple :

\begin{enumerate}
 \item Nous allons récupérer la position des caractères dans notre alphabet
 \item Nous allons alors multiplier/additionner/soustraire (modulo un nombre) cette position par un coefficient donné (la clé)
 \item La nouvelle position trouvée donne le chiffre du caractère clair
\end{enumerate}
Nous pouvons noter cet algorithme de la manière suivante :
\begin{center}
    \begin{equation}
	E(M)\equiv (kM+r)[n]
    \end{equation}
\end{center}
Où $k$ et $r$ sont des entiers et $k$ est premier par rapport à $n$ (le modulo utilisé, la taille de l'alphabet).\\
Pour déchiffrer un message ainsi codé, nous pouvons alors utiliser l'équation suivante :
\begin{center}
    \begin{equation}
	D(C)\equiv ((n-k)C+(n-r))[n]
    \end{equation}
\end{center}
Un des inconvénients de ce système est qu'il ne change pas la fréquence d'apparition des lettres, nous avons ici un système à permutations. Il est alors simple de reconnaître, en connaissant la langue d'origine du texte les fréquences les plus élevées. Une fois ces fréquences trouvées, nous pouvons poser deux équations à deux inconnues pour retrouver les coefficients $k$ et $t$.\\
Si, par exemple, nous avons un texte en anglais chiffré ainsi avec comme lettres les plus présentes les lettres F et C, nous pouvons en déduire que le F correspond à la lettre E (lettre la plus utilisée en anglais) et que le C correspond à la lettre T (seconde lettre la plus utilisée en anglais.\\
Nous pouvons alors assumer que $E(4)=5$ et $E(19)=2$. Nous pouvons alors retrouver les deux équations suivantes :
\begin{center}
    \begin{equation}
	\left\{
	\begin{array}{lll}
	    4k+r & \equiv & 5[26] \\
	    19k+r & \equiv & 2[26]
	\end{array}
	\right.
    \end{equation}
\end{center}
La résolution de ce système nous permet de trouver que $k=5$ et $r=11$. Nous avons ainsi les équations suivantes liées au chiffrement et déchiffrement :
\begin{center}
    \begin{equation}
	E(M)\equiv (5M+11)[26]
    \end{equation}
    \begin{equation}
	D(C)\equiv (21C+15)[26]
    \end{equation}
\end{center}
Nous pouvons alors retrouver le texte clair.\\~\\\par
Dans notre cas, nous avons un système légèrement différent. En effet, ce dernier (c.f. \ref{scext1}) comporte trois fonctions qui seront utilisées tour-à-tour en fonction de l'indice du bit à chiffrer.\\
Connaissant l'algorithme utilisé, nous savons que le coefficient $k$ est ici égal à $1$. La valeur de $r$ va quand à elle varier en fonction de la fonction utilisée ($f_1, f_2 \mbox{ ou } f_3$)
\subsection{\textit{SCEX\_T2}}
Contrairement à l'algorithme précédent, \textit{SCEX\_T2} ne va pas utiliser les fonctions $f_1, f_2 \mbox{ et } f_3$ tour à tour mais toutes en même temps lors de la partie de chiffrement. Nous allons ici nous retrouver devant la formule suivante :
\begin{center}
    \begin{equation}
	c_t=m_t\oplus\left(f_1\left(x_3,x_2,x_1\right)\oplus f_2\left(x_3,x_2,x_1\right)\oplus f_3\left(x_3,x_2,x_1\right) \right)
    \end{equation}
\end{center}
La valeur du coefficient $k$ reste ici encore égale à $1$ mais le coefficient $r$ ne va pas se calculer de la même manière. En effet, ce dernier n'est plus dépendant d'une seule fonction par tour mais d'un XOR des trois fonctions à chaque tour.